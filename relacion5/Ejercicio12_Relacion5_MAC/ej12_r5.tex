\documentclass{article}
\usepackage[utf8]{inputenc}
\usepackage{listings}

\title{Ejercicio 11 Relación 5}
\author{---- \\ ---}
\date{}

\begin{document}

\maketitle

Consideremos el problema anterior, pero ahora con dos tipos de paréntesis (, ) y [, ]. Demostrar que determinar si están bien escritos también está en L. Aquı́ ([]()) está permitido, pero ([)] no. Demostrar que si la entrada se lee de izquierda a derecha sin volver hacia atrás, requieren O(n) de memoria.\\
\\
Este problema se puede resolver con una MT determinista con tres cintas:
\begin{itemize}
    \item 1º cinta: Entrada. Solo leerme en esta cinta.
    \item 2º cinta: Al principio vacía, la usaremos de contador, que empezará a 0.
\end{itemize}

Cada vez que encontremos `(' en la cinta sumamos 1 a la segunda. Cada vez que encontremos `)' restamos 1 a la cinta, pero antes de restar, comprobamos si es 0. Si es 0 y vamos a restar 1, no están emparejados y por tanto, rechazamos.\\
Si llegamos al final de la 1º cinta y la 2º cinta contiene un numero distinto a 0 rechazamos, si es igual a 0 aceptamos.

\section{Máquina de Turing}
\begin{lstlisting}
    Mientras el simbolo de la cinta 1 != #
        Si el simbolo de la cinta 1 == `('
            Sumamos 1 en la cinta 2
        Si no
            Si la cinta 2 == 0
                Rechazamos
            Si no
                Restamos 1 en la cinta 2
    Si la cinta 2 == 0
        Aceptamos
    Si no
        Rechazamos
\end{lstlisting}

Como solo escribimos en la 2º cinta, solo tendremos en cuenta esto. Nos damos cuenta que a lo sumo vamos a contar cuantos símbolos tiene la entrada ( el numero más grande que escribimos es el numero de `(', luego para la entrada `((((' estamos contando su longitud) y sabemos que para escribir un número x necesitamos $log(x)$ espacio, en el caso de usar binario $log_{2}(x)$, luego el problema esta en L.

\end{document}
