\documentclass{article}
\usepackage[utf8]{inputenc}
\usepackage{listings}
\usepackage{vmargin}

\title{Entrega Relación 5}
\author{Antonio David Villegas Yeguas}
\date{}

\setmarginsrb{2 cm}{1 cm}{2 cm}{2 cm}{1 cm}{1.5 cm}{1 cm}{1.5 cm}

\begin{document}

\maketitle


\section*{Ejercicio 1: Demostrar que NL es cerrado para la clausura de Kleene.}

Para este ejercicio se nos pide demostrar que si un lenguaje $L$ está en la clase $NL$ entonces $L^*$ está también en la clase $NL$.\\

Supongamos que tenemos un lenguaje $L = \{0, 1\}$ en binario.

Para demostrar que NL es cerrado para la clausura de Kleene usaremos el problema de existencia de un camino en un grafo $G$ dirigido. Este problema dice que dado un grafo dirigido con la siguiente representación:

$$ (x_1, y_1)|...|(x_m, y_m) $$

Siendo $x_i$ y $y_i$ vértices numerados en binario. El problema es si dado dos vértices $s$ y $t$, existe un camino de longitud $l$ desde $s$ hasta $t$ en $G$. \\


Este problema sabemos que es NL-Completo, y además vemos como todos los caminos que se pueden formar representan las cadenas formadas por $L^*$, luego como tenemos un lenguaje $L$ que se reconoce en $NL$, cualquier palabra de $L^*$, como hemos demostrado con este problema, se puede reconocer también en NL, haciendo NL cerrado para la clausura de Kleene.

\section*{Ejercicio 2: Demostrar que todo lenguaje regular está en L.}

	Al ser un lenguaje regular, sabemos que este es aceptado por un Autómata Finito Determinista. \\

Dichos autómatas no utilizan memoria, e incluso podemos hacer una MT determinista que simule a dicho autómata, haciendo que solo se mueva a la derecha, tenga el mismo número de estados que el AFD y que por cada transición del AFD tenemos una transición de nuestra MT que se mueve a la derecha, y la transición entre estados sea equivalente (cuando el AFD pasa a un estado $q_i$ nuestra MT pasará al estado $q_i$ equivalente) y lo más importante, no escribe en ninguna otra casilla, por lo que solamente lee símbolos de la cinta de entrada.\\

De esta forma, al solo leer símbolos de la entrada, y los únicos símbolos para comprobar que está en espacio $L$ son los nuevos escritos, esta MT se encuentra en espacio $L$, y por lo tanto todo lenguaje regular está en $L$.

\section*{Ejercicio 3:}

\Large{\textbf{Sea $MULT = \{a * b * c$ tales que a, b y c son números naturales en binario y $a  b = c\}$. Demostrar que $MULT \in L$}}
\\

\normalsize
Básicamente comprobar que $n + m < log_2(2n + 2m)$, siendo $n = |a|$ y $m = |b|$, ya que sabemos que la multiplicación como mucho es de longitud $n + l$ y en la cinta de datos tenemos $2n + 2m$ a lo mucho.

\end{document}
