\documentclass{article}
\usepackage[utf8]{inputenc}
\usepackage{listings}
\usepackage{vmargin}

\title{Entrega Relación 5}
\author{Antonio David Villegas Yeguas}
\date{}

\setmarginsrb{2 cm}{1 cm}{2 cm}{2 cm}{1 cm}{1.5 cm}{1 cm}{1.5 cm}

\begin{document}

\maketitle


\section*{Ejercicio 1: Demostrar que NL es cerrado para la clausura de Kleene.}

Para este ejercicio se nos pide demostrar que si un lenguaje $L$ está en la clase $NL$ entonces $L^*$ está también en la clase $NL$.

\section*{Ejercicio 2: Demostrar que todo lenguaje regular está en L.}

Para aceptar o rechazar un lenguaje regular no tienes que escribir, por lo que al usar 0 espacio todo lenguaje regular está en L.

\section*{Ejercicio 3:}

\Large{\textbf{Sea $MULT = \{a * b * c$ tales que a, b y c son números naturales en binario y $a  b = c\}$. Demostrar que $MULT \in L$}}
\\

\normalsize
Básicamente comprobar que $n + m < log_2(2n + 2m)$, siendo $n = |a|$ y $m = |b|$, ya que sabemos que la multiplicación como mucho es de longitud $n + l$ y en la cinta de datos tenemos $2n + 2m$ a lo mucho.

\end{document}
